\documentclass[a4paper,12pt]{article}
\usepackage{mathtools,amsfonts,amssymb,amsmath, bm,commath,multicol}
\usepackage{algorithmicx, tkz-graph, algorithm, fancyhdr, pgfplots}
\usepackage{fancyvrb}

\usepackage[noend]{algpseudocode}

\pagestyle{fancy}
\fancyhf{}
\rhead{21/2/2017 ::: Group yadadya}
\lhead{15F020 ::: Problemset 3}
\rfoot{\thepage}

\DefineVerbatimEnvironment{juliaout}{Verbatim}{}
\DefineVerbatimEnvironment{juliacode}{Verbatim}{fontshape=sl, fontsize=\tiny}
\DefineVerbatimEnvironment{juliaterm}{Verbatim}{}


\begin{document}


\section{Pricing and Hedging Asian Options}



\subsection{}

\subsubsection{}

Differentiating the discounted portfolio value with respect to time:
\begin{align}
\tilde{V}_t &= e^{-rt}V_t \\
\frac{d}{d_t}\tilde{V}_t &= -r \tilde{V}_t + e^{-rt} ( \frac{d}{d_t}V_t)
\end{align}
We apply It\^{o}'s formula to find the derivative with respect to time of our portfolio value. Plugging in the definitions of $dY_t$, our asian option price, and $dS_t$, our stock price which follows the Black-Scholes. We use $v = v(t, S_t, Y_t)$ for ease of notation for the value function:
%
\begin{align*}
\frac{d}{d_t}V_t & = \frac{d}{d_t}v + \frac{dY_t}{d_t}\frac{d}{d_y}v + \frac{dS_t}{d_t}\frac{d}{d_x}v + \frac{1}{2}\frac{d^2}{d_{xx}}v \\
\frac{d}{d_t}V_t &= \frac{d}{d_t}v + \frac{S_td_t}{d_t}\frac{d}{d_y}v + \frac{S_trd_t + S_t \sigma dW_t}{d_t}\frac{d}{d_x}v + \frac{1}{2}\sigma^2 S_t^2 \frac{d^2}{d_{xx}}v\\
\frac{d}{d_t}V_t &= \frac{d}{d_t}v + S_t \frac{d}{d_y}v + S_tr\frac{d}{d_x}v + S_t \sigma \frac{d}{d_t}dW_t \frac{d}{d_x}v + \frac{1}{2}\sigma^2 S_t^2 \frac{d^2}{d_{xx}}v\\
\end{align*}
%
Simplifying notation by using $\partial_t = \frac{d}{d_t}$ and $x = S_t$:
\begin{align}
\partial_t V_t &= \partial_t V_t + x \partial_y V_t + rx \partial_x V_t + S_t \sigma \partial_tW_t \partial_xV_t + \frac{1}{2}\sigma^2 S_t^2 \partial_{xx}^2 V_t
\end{align}
%
Substituting (3) into (2) and setting the change in discounted portfolio value to zero to satisfy the self-financing condition:
%
\begin{align*}
0 &= -r \tilde{V}_t + e^{-rt} ( \frac{d}{d_t}V_t) \\
r \tilde{V}_t e^{rt} &= \partial_t v + x \partial_y v + rx \partial_x v + S_t \sigma \partial_tW_t \partial_x v + \frac{1}{2}\sigma^2 S_t^2 \partial_{xx}^2 v
\end{align*}
%
Taking advantage of the martingale property and (1) for the portfolio value:
\begin{align*}
rV_t &= \partial_t v(t, S_t, Y_t) + x \partial_y v(t, S_t, Y_t) + rx \partial_x v(t, S_t, Y_t)  + \frac{1}{2}\sigma^2 S_t^2 \partial_{xx}^2 v(t, S_t, Y_t)
\end{align*}


\subsubsection{Boundary Conditions}
We begin by more fully parameterizing our Y function and making use of the fact that our asset price is the result of geometric brownian motion steps:
\begin{align*}
Y_{T} &= \int_0^T S_udu \\
Y_{t, x, T} &= \int_t^T x \int_t^u W_x dx \ du
\end{align*}
We will re-write our payoff as a function of two variables by utilizing the fact that Y is an integral, and as such can be decomposed into the sum of two parts:
\begin{align*}
H(Y_T) &= \bigg( \frac{1}{T}Y_T - K \bigg)^+ \\
H(Y_{0, S_0, t}, Y_{t, S_t, T}) &= \bigg( \frac{1}{T}Y_{0,S_0,t} +  \frac{1}{T}Y_{t, S_t, T} - K \bigg)^+
\end{align*}
This allows us to use the fact that $Y_{0, S_0, t}$ is $\mathcal{F}_t$ measurable, $S_t$ is $\mathcal{F}_t$ is measurable, and $Y_{t, s, T}$ is independent of $\mathcal{F}_t$:
\begin{align*}
\tilde{V_t} &= e^{-r(T-t)} \ \mathbb{E} \big[ H (y, Y_{t, x, T}) | _{x = S_t, y = Y_t} \big]
\end{align*}
And we solve with $y = Y_t$ and $S_t = x$ for our boundary conditions. The first when $x = 0$, it is easy to see that:
$$
Y_{t, 0, T} = \int_t^T 0 \int_t^u W_x dx \ du  = 0
$$
Therefore:
\begin{align*}
v(t, 0, y) &= e^{-r(T-t)} \ \mathbb{E} \big[ \bigg( \frac{y}{T} +  \frac{Y_{t,0,T}}{T} - K \bigg)^+ | _{x = 0, y = y} \big] \\
v(t, 0, y) &= e^{-r(T-t)} \ \mathbb{E} \big[ \bigg( \frac{y}{T} - K \bigg)^+ | _{x = 0, y = y} \big] \\
v(t, 0, y) &= e^{-r(T-t)} \ \bigg( \frac{y}{T} - K \bigg)^+  \\
\end{align*}
Similarly for the condition when $t = T$, the second parameter of our payoff function is the integral over 0 distance and therefore is equal to 0:
\begin{align*}
v(T, x, y) &= e^{-r(T-t)} \ \mathbb{E} \big[ \bigg( \frac{y}{T} +  \frac{Y_{T, x, T}}{T} - K \bigg)^+ | _{x = x, y = y} \big] \\
v(T, x, y) &= e^{-r(T-t)} \ \mathbb{E} \big[ \bigg( \frac{y}{T} +  0  - K \bigg)^+ | _{x = x, y = y} \big] \\
v(T, x, y) &= e^{-r(T-t)} \ \bigg( \frac{y}{T} - K \bigg)^+ \\
\end{align*}


\end{document}